\documentclass[spanish]{beamer}

\usetheme{default}

\usepackage{polyglossia}

\setdefaultlanguage{spanish}

\usepackage{fancyvrb}

\DefineShortVerb{\|}
\DefineVerbatimEnvironment{code}{Verbatim}{frame=lines}

\title{Haskell práctico}
\subtitle{Analizadores}
\author{Juan Pedro Villa Isaza}
\institute{Stack Builders}
\date{\today}

\logo{\includegraphics[scale=0.15]{../images/stackbuilders-logo.png}}

\begin{document}

%%%%%%%%%%%%%%%%%%%%%%%%%%%%%%%%%%%%%%%%%%%%%%%%%%%%%%%%%%%%%%%%%%%%%%

\frame{\titlepage}

%%%%%%%%%%%%%%%%%%%%%%%%%%%%%%%%%%%%%%%%%%%%%%%%%%%%%%%%%%%%%%%%%%%%%%

\begin{frame}
  \frametitle{Haskell es...}

  \begin{itemize}
  \item
    Funcional
  \item
    Puro
  \item
    \emph{Multipropósito}
  \item
    ...
  \end{itemize}
\end{frame}

%%%%%%%%%%%%%%%%%%%%%%%%%%%%%%%%%%%%%%%%%%%%%%%%%%%%%%%%%%%%%%%%%%%%%%

\begin{frame}[fragile]
  \frametitle{Haskell es funcional y puro}
  \framesubtitle{Fibonacci}

  \begin{code}
fibonacci :: Integer -> Integer
fibonacci 0 = 0
fibonacci 1 = 1
fibonacci n = fibonacci (n - 1) + fibonacci (n - 2)
  \end{code}
\end{frame}

%%%%%%%%%%%%%%%%%%%%%%%%%%%%%%%%%%%%%%%%%%%%%%%%%%%%%%%%%%%%%%%%%%%%%%

\begin{frame}[fragile]
  \frametitle{Haskell es funcional y puro}
  \framesubtitle{Fibonacci}

  \begin{code}
> fibonacci 5
5
  \end{code}
\end{frame}


%%%%%%%%%%%%%%%%%%%%%%%%%%%%%%%%%%%%%%%%%%%%%%%%%%%%%%%%%%%%%%%%%%%%%%

\begin{frame}[fragile]
  \frametitle{Haskell es funcional y puro}
  \framesubtitle{Fibonacci (C)}

  \begin{code}
int fibonacci(int n) {
  if (n == 0)
    return 0;
  else if (n == 1)
    return 1;
  else
    return fibonacci (n - 1) + fibonacci (n - 2);
}
  \end{code}
\end{frame}

%%%%%%%%%%%%%%%%%%%%%%%%%%%%%%%%%%%%%%%%%%%%%%%%%%%%%%%%%%%%%%%%%%%%%%

\begin{frame}[fragile]
  \frametitle{Haskell es funcional y puro}
  \framesubtitle{Fibonacci (C)}

  \begin{code}
int main() {
  printf("%d\n", fibonacci(5));
}
  \end{code}
\end{frame}

%%%%%%%%%%%%%%%%%%%%%%%%%%%%%%%%%%%%%%%%%%%%%%%%%%%%%%%%%%%%%%%%%%%%%%

\begin{frame}[fragile]
  \frametitle{Haskell es funcional y puro}
  \framesubtitle{Fibonacci (C)}

  Los diez primeros números de Fibonacci:
  \begin{code}
for (int i = 1; i <= 10; i++) {
  printf("%d\n", fibonacci(i - 1));
}
  \end{code}
\end{frame}

%%%%%%%%%%%%%%%%%%%%%%%%%%%%%%%%%%%%%%%%%%%%%%%%%%%%%%%%%%%%%%%%%%%%%%

\begin{frame}[fragile]
  \frametitle{Haskell es funcional y puro}
  \framesubtitle{Fibonacci}

  Los diez primeros números de Fibonacci:
  \begin{code}
> map fibonacci [0..9]
[0,1,1,2,3,5,8,13,21,34]
  \end{code}
\end{frame}

%%%%%%%%%%%%%%%%%%%%%%%%%%%%%%%%%%%%%%%%%%%%%%%%%%%%%%%%%%%%%%%%%%%%%%

\begin{frame}[fragile]
  \frametitle{Haskell es funcional y puro}
  \framesubtitle{Fibonacci}

  Todos los números de Fibonacci:
  \begin{code}
> let fibonaccis = map fibonacci [0..]
  \end{code}
\end{frame}

%%%%%%%%%%%%%%%%%%%%%%%%%%%%%%%%%%%%%%%%%%%%%%%%%%%%%%%%%%%%%%%%%%%%%%

\begin{frame}[fragile]
  \frametitle{Haskell es funcional y puro}
  \framesubtitle{Fibonacci}

  Los diez primeros números de Fibonacci:
  \begin{code}
> take 10 fibonaccis
[0,1,1,2,3,5,8,13,21,34]
  \end{code}
\end{frame}

%%%%%%%%%%%%%%%%%%%%%%%%%%%%%%%%%%%%%%%%%%%%%%%%%%%%%%%%%%%%%%%%%%%%%%

\begin{frame}[fragile]
  \frametitle{Haskell es funcional y puro}
  \framesubtitle{Fibonacci (C)}

  Los diez primeros números pares de Fibonacci:
  \begin{code}
for (int i = 1, j = 0; i <= 10; j++) {
  if (fibonacci(j) % 2 == 0) {
    printf("%d\n", fibonacci(j));
    i = i + 1;
  }
}
  \end{code}
\end{frame}

%%%%%%%%%%%%%%%%%%%%%%%%%%%%%%%%%%%%%%%%%%%%%%%%%%%%%%%%%%%%%%%%%%%%%%

\begin{frame}[fragile]
  \frametitle{Haskell es funcional y puro}
  \framesubtitle{Fibonacci}

  Los diez primeros números pares de Fibonacci:
  \begin{code}
> take 10 (filter even fibonaccis)
[0,2,8,34,144,610,2584,10946,46368,196418]
  \end{code}
\end{frame}

%%%%%%%%%%%%%%%%%%%%%%%%%%%%%%%%%%%%%%%%%%%%%%%%%%%%%%%%%%%%%%%%%%%%%%

\begin{frame}[fragile]
  \frametitle{Haskell es funcional y puro}
  \framesubtitle{Fibonacci (C)}

  La suma de los diez primeros números pares de Fibonacci:
  \begin{code}
int s = 0;
for (int i = 1, j = 0; i <= 10; j++) {
  if (fibonacci(j) % 2 == 0) {
    s = s + fibonacci(j);
    i = i + 1;
  }
}
printf("%d\n", s);
  \end{code}
\end{frame}

%%%%%%%%%%%%%%%%%%%%%%%%%%%%%%%%%%%%%%%%%%%%%%%%%%%%%%%%%%%%%%%%%%%%%%

\begin{frame}[fragile]
  \frametitle{Haskell es funcional y puro}
  \framesubtitle{Fibonacci}

  La suma de los diez primeros números pares de Fibonacci:
  \begin{code}
> sum (take 10 (filter even fibonaccis))
257114
  \end{code}
\end{frame}

%%%%%%%%%%%%%%%%%%%%%%%%%%%%%%%%%%%%%%%%%%%%%%%%%%%%%%%%%%%%%%%%%%%%%%

\begin{frame}
  \frametitle{Haskell es multipropósito}
  \framesubtitle{Analizadores (Nombres de usuario)}

  \texttt{¡Hola, @\alert<2>{stackbuilders}! \#haskell \#cuenca}
\end{frame}

%%%%%%%%%%%%%%%%%%%%%%%%%%%%%%%%%%%%%%%%%%%%%%%%%%%%%%%%%%%%%%%%%%%%%%

\begin{frame}[fragile]
  \frametitle{Haskell es multipropósito}
  \framesubtitle{Analizadores (megaparsec-4.4.0)}

  \begin{code}
import Text.Megaparsec
import Text.Megaparsec.String (Parser)
  \end{code}
\end{frame}

%%%%%%%%%%%%%%%%%%%%%%%%%%%%%%%%%%%%%%%%%%%%%%%%%%%%%%%%%%%%%%%%%%%%%%

\begin{frame}[fragile]
  \frametitle{Haskell es multipropósito}
  \framesubtitle{Analizadores (megaparsec-4.4.0)}

  \begin{code}
> parseTest (char '@') "@stackbuilders"
'@'
  \end{code}
\end{frame}

%%%%%%%%%%%%%%%%%%%%%%%%%%%%%%%%%%%%%%%%%%%%%%%%%%%%%%%%%%%%%%%%%%%%%%

\begin{frame}[fragile]
  \frametitle{Haskell es multipropósito}
  \framesubtitle{Analizadores (megaparsec-4.4.0)}

  \begin{code}
> parseTest (char '@') "#stackbuilders"
1:1:
unexpected '#'
expecting '@'
  \end{code}
\end{frame}

%%%%%%%%%%%%%%%%%%%%%%%%%%%%%%%%%%%%%%%%%%%%%%%%%%%%%%%%%%%%%%%%%%%%%%

\begin{frame}[fragile]
  \frametitle{Haskell es multipropósito}
  \framesubtitle{Analizadores (megaparsec-4.4.0)}

  \begin{code}
> parseTest (some alphaNumChar) ""
1:1:
unexpected end of input
expecting alphanumeric character
  \end{code}
\end{frame}

%%%%%%%%%%%%%%%%%%%%%%%%%%%%%%%%%%%%%%%%%%%%%%%%%%%%%%%%%%%%%%%%%%%%%%

\begin{frame}[fragile]
  \frametitle{Haskell es multipropósito}
  \framesubtitle{Analizadores (megaparsec-4.4.0)}

  \begin{code}
> parseTest (some alphaNumChar) "cuenca"
"cuenca"
  \end{code}
\end{frame}

%%%%%%%%%%%%%%%%%%%%%%%%%%%%%%%%%%%%%%%%%%%%%%%%%%%%%%%%%%%%%%%%%%%%%%

\begin{frame}[fragile]
  \frametitle{Haskell es multipropósito}
  \framesubtitle{Analizadores (megaparsec-4.4.0)}

  \begin{code}
> parseTest (many alphaNumChar) ""
""
  \end{code}
\end{frame}

%%%%%%%%%%%%%%%%%%%%%%%%%%%%%%%%%%%%%%%%%%%%%%%%%%%%%%%%%%%%%%%%%%%%%%

\begin{frame}[fragile]
  \frametitle{Haskell es multipropósito}
  \framesubtitle{Analizadores (megaparsec-4.4.0)}

  \begin{code}
> parseTest (many alphaNumChar) "cuenca"
"cuenca"
  \end{code}
\end{frame}

%%%%%%%%%%%%%%%%%%%%%%%%%%%%%%%%%%%%%%%%%%%%%%%%%%%%%%%%%%%%%%%%%%%%%%

\begin{frame}
  \frametitle{Haskell es multipropósito}
  \framesubtitle{Analizadores (Nombres de usuario)}

  \texttt{¡Hola, @\alert{stackbuilders}! \#haskell \#cuenca}
\end{frame}


%%%%%%%%%%%%%%%%%%%%%%%%%%%%%%%%%%%%%%%%%%%%%%%%%%%%%%%%%%%%%%%%%%%%%%

\begin{frame}[fragile]
  \frametitle{Haskell es multipropósito}
  \framesubtitle{Analizadores (Nombres de usuario)}

  \begin{code}
username :: Parser String
username = do
  _ <- char '@'
  some alphaNumChar
  \end{code}
\end{frame}

%%%%%%%%%%%%%%%%%%%%%%%%%%%%%%%%%%%%%%%%%%%%%%%%%%%%%%%%%%%%%%%%%%%%%%

\begin{frame}[fragile]
  \frametitle{Haskell es multipropósito}
  \framesubtitle{Analizadores (Nombres de usuario)}

  \begin{code}
> parseTest username "@stackbuilders"
"stackbuilders"
  \end{code}
\end{frame}

%%%%%%%%%%%%%%%%%%%%%%%%%%%%%%%%%%%%%%%%%%%%%%%%%%%%%%%%%%%%%%%%%%%%%%

\begin{frame}[fragile]
  \frametitle{Haskell es multipropósito}
  \framesubtitle{Analizadores (Nombres de usuario)}

  \begin{code}
> parseTest username "¡Hola, @stackbuilders!"
1:1:
unexpected '¡'
expecting '@'
  \end{code}
\end{frame}

%%%%%%%%%%%%%%%%%%%%%%%%%%%%%%%%%%%%%%%%%%%%%%%%%%%%%%%%%%%%%%%%%%%%%%

\begin{frame}[fragile]
  \frametitle{Haskell es multipropósito}
  \framesubtitle{Analizadores (Nombres de usuario)}

  \begin{code}
usernames :: Parser [String]
usernames = do
  _ <- many (noneOf "@")
  (:) <$> username <*> usernames <|> return []
  \end{code}
\end{frame}

%%%%%%%%%%%%%%%%%%%%%%%%%%%%%%%%%%%%%%%%%%%%%%%%%%%%%%%%%%%%%%%%%%%%%%

\begin{frame}[fragile]
  \frametitle{Haskell es multipropósito}
  \framesubtitle{Analizadores (Nombres de usuario)}

  \begin{code}
> parseTest usernames "¡Hola, @stackbuilders!"
["stackbuilders"]
  \end{code}
\end{frame}

%%%%%%%%%%%%%%%%%%%%%%%%%%%%%%%%%%%%%%%%%%%%%%%%%%%%%%%%%%%%%%%%%%%%%%

\begin{frame}[fragile]
  \frametitle{Haskell es multipropósito}
  \framesubtitle{Analizadores (Nombres de usuario)}

  \begin{code}
> parseTest usernames "¡Hola, Stack Builders!"
[]
  \end{code}
\end{frame}

%%%%%%%%%%%%%%%%%%%%%%%%%%%%%%%%%%%%%%%%%%%%%%%%%%%%%%%%%%%%%%%%%%%%%%

\begin{frame}
  \frametitle{Haskell es multipropósito}
  \framesubtitle{Analizadores (Etiquetas)}

  \texttt{¡Hola, @stackbuilders! \#\alert<2>{haskell} \#\alert<2>{cuenca}}
\end{frame}

%%%%%%%%%%%%%%%%%%%%%%%%%%%%%%%%%%%%%%%%%%%%%%%%%%%%%%%%%%%%%%%%%%%%%%


\begin{frame}[fragile]
  \frametitle{Haskell es multipropósito}
  \framesubtitle{Analizadores (Etiquetas)}

  \begin{code}
hashtag :: Parser String
hashtag = do
  _ <- char '#'
  some alphaNumChar
  \end{code}
\end{frame}

%%%%%%%%%%%%%%%%%%%%%%%%%%%%%%%%%%%%%%%%%%%%%%%%%%%%%%%%%%%%%%%%%%%%%%

\begin{frame}[fragile]
  \frametitle{Haskell es multipropósito}
  \framesubtitle{Analizadores (Etiquetas)}

  \begin{code}
> parseTest hashtag "#haskell"
"haskell"
  \end{code}
\end{frame}

%%%%%%%%%%%%%%%%%%%%%%%%%%%%%%%%%%%%%%%%%%%%%%%%%%%%%%%%%%%%%%%%%%%%%%

\begin{frame}[fragile]
  \frametitle{Haskell es multipropósito}
  \framesubtitle{Analizadores (Etiquetas)}

  \begin{code}
> parseTest hashtag "#haskell #cuenca"
"haskell"
  \end{code}
\end{frame}

%%%%%%%%%%%%%%%%%%%%%%%%%%%%%%%%%%%%%%%%%%%%%%%%%%%%%%%%%%%%%%%%%%%%%%

\begin{frame}[fragile]
  \frametitle{Haskell es multipropósito}
  \framesubtitle{Analizadores (Etiquetas)}

  \begin{code}
hashtags :: Parser [String]
hashtags = do
  _ <- many (noneOf "#")
  (:) <$> hashtag <*> hashtags <|> return []
  \end{code}
\end{frame}

%%%%%%%%%%%%%%%%%%%%%%%%%%%%%%%%%%%%%%%%%%%%%%%%%%%%%%%%%%%%%%%%%%%%%%

\begin{frame}[fragile]
  \frametitle{Haskell es multipropósito}
  \framesubtitle{Analizadores (Etiquetas)}

  \begin{code}
> parseTest hashtags "#haskell #cuenca"
["haskell","cuenca"]
  \end{code}
\end{frame}

%%%%%%%%%%%%%%%%%%%%%%%%%%%%%%%%%%%%%%%%%%%%%%%%%%%%%%%%%%%%%%%%%%%%%%

\begin{frame}
  \frametitle{Haskell es multipropósito}
  \framesubtitle{Analizadores (CSV)}

  \texttt{aaa,bbb,ccc}\\
  \texttt{xxx,yyy,zzz}
\end{frame}

%%%%%%%%%%%%%%%%%%%%%%%%%%%%%%%%%%%%%%%%%%%%%%%%%%%%%%%%%%%%%%%%%%%%%%

\begin{frame}[fragile]
  \frametitle{Haskell es multipropósito}
  \framesubtitle{Analizadores (CSV)}

  RFC 4180:
  \begin{code}
file = record *(CRLF record) CRLF

record = field *(COMMA field)

field = *TEXTDATA
  \end{code}
\end{frame}

%%%%%%%%%%%%%%%%%%%%%%%%%%%%%%%%%%%%%%%%%%%%%%%%%%%%%%%%%%%%%%%%%%%%%%

\begin{frame}[fragile]
  \frametitle{Haskell es multipropósito}
  \framesubtitle{Analizadores (CSV)}

  \begin{code}
field = *TEXTDATA
  \end{code}
\end{frame}

%%%%%%%%%%%%%%%%%%%%%%%%%%%%%%%%%%%%%%%%%%%%%%%%%%%%%%%%%%%%%%%%%%%%%%

\begin{frame}[fragile]
  \frametitle{Haskell es multipropósito}
  \framesubtitle{Analizadores (CSV)}

  \begin{code}
field :: Parser String
field = many (noneOf ",\n")
  \end{code}
\end{frame}

%%%%%%%%%%%%%%%%%%%%%%%%%%%%%%%%%%%%%%%%%%%%%%%%%%%%%%%%%%%%%%%%%%%%%%

\begin{frame}[fragile]
  \frametitle{Haskell es multipropósito}
  \framesubtitle{Analizadores (CSV)}

  \begin{code}
> parseTest field "aaa"
"aaa"
  \end{code}
\end{frame}

%%%%%%%%%%%%%%%%%%%%%%%%%%%%%%%%%%%%%%%%%%%%%%%%%%%%%%%%%%%%%%%%%%%%%%

\begin{frame}[fragile]
  \frametitle{Haskell es multipropósito}
  \framesubtitle{Analizadores (CSV)}

  \begin{code}
> parseTest field "bbb"
"bbb"
  \end{code}
\end{frame}

%%%%%%%%%%%%%%%%%%%%%%%%%%%%%%%%%%%%%%%%%%%%%%%%%%%%%%%%%%%%%%%%%%%%%%

\begin{frame}[fragile]
  \frametitle{Haskell es multipropósito}
  \framesubtitle{Analizadores (CSV)}

  \begin{code}
> parseTest field "ccc"
"ccc"
  \end{code}
\end{frame}

%%%%%%%%%%%%%%%%%%%%%%%%%%%%%%%%%%%%%%%%%%%%%%%%%%%%%%%%%%%%%%%%%%%%%%

\begin{frame}[fragile]
  \frametitle{Haskell es multipropósito}
  \framesubtitle{Analizadores (CSV)}

  \begin{code}
record = field *(COMMA field)
  \end{code}
\end{frame}

%%%%%%%%%%%%%%%%%%%%%%%%%%%%%%%%%%%%%%%%%%%%%%%%%%%%%%%%%%%%%%%%%%%%%%

\begin{frame}[fragile]
  \frametitle{Haskell es multipropósito}
  \framesubtitle{Analizadores (CSV)}

  \begin{code}
record :: Parser [String]
record = sepBy field (char ',')
  \end{code}
\end{frame}

%%%%%%%%%%%%%%%%%%%%%%%%%%%%%%%%%%%%%%%%%%%%%%%%%%%%%%%%%%%%%%%%%%%%%%

\begin{frame}[fragile]
  \frametitle{Haskell es multipropósito}
  \framesubtitle{Analizadores (CSV)}

  \begin{code}
> parseTest record "aaa,bbb,ccc"
["aaa","bbb","ccc"]
  \end{code}
\end{frame}

%%%%%%%%%%%%%%%%%%%%%%%%%%%%%%%%%%%%%%%%%%%%%%%%%%%%%%%%%%%%%%%%%%%%%%

\begin{frame}[fragile]
  \frametitle{Haskell es multipropósito}
  \framesubtitle{Analizadores (CSV)}

  \begin{code}
> parseTest record "xxx,yyy,zzz"
["xxx","yyy","zzz"]
  \end{code}
\end{frame}

%%%%%%%%%%%%%%%%%%%%%%%%%%%%%%%%%%%%%%%%%%%%%%%%%%%%%%%%%%%%%%%%%%%%%%

\begin{frame}[fragile]
  \frametitle{Haskell es multipropósito}
  \framesubtitle{Analizadores (CSV)}

  \begin{code}
file = record *(CRLF record) CRLF
  \end{code}
\end{frame}

%%%%%%%%%%%%%%%%%%%%%%%%%%%%%%%%%%%%%%%%%%%%%%%%%%%%%%%%%%%%%%%%%%%%%%

\begin{frame}[fragile]
  \frametitle{Haskell es multipropósito}
  \framesubtitle{Analizadores (CSV)}

  \begin{code}
file :: Parser [[String]]
file = endBy1 record eol
  \end{code}
\end{frame}

%%%%%%%%%%%%%%%%%%%%%%%%%%%%%%%%%%%%%%%%%%%%%%%%%%%%%%%%%%%%%%%%%%%%%%

\begin{frame}[fragile]
  \frametitle{Haskell es multipropósito}
  \framesubtitle{Analizadores (CSV)}

  \begin{code}
> parseTest file "aaa,bbb,ccc\nxxx,yyy,zzz\n"
[["aaa","bbb","ccc"],["xxx","yyy","zzz"]]
  \end{code}
\end{frame}

%%%%%%%%%%%%%%%%%%%%%%%%%%%%%%%%%%%%%%%%%%%%%%%%%%%%%%%%%%%%%%%%%%%%%%

\begin{frame}[fragile]
  \frametitle{Haskell es multipropósito}
  \framesubtitle{Analizadores (CSV)}

  \begin{code}
file :: Parser [[String]]
file = endBy1 record eol

record :: Parser [String]
record = sepBy field (char ',')

field :: Parser String
field = many (noneOf ",\n")
  \end{code}
\end{frame}

%%%%%%%%%%%%%%%%%%%%%%%%%%%%%%%%%%%%%%%%%%%%%%%%%%%%%%%%%%%%%%%%%%%%%%

\begin{frame}[fragile]
  \frametitle{Haskell es multipropósito}
  \framesubtitle{Analizadores (CSV)}

  \begin{code}
file = record *(CRLF record) CRLF
file = endBy1 record eol

record = field *(COMMA field)
record = sepBy field (char ',')

field = *TEXTDATA
field = many (noneOf ",\n")
  \end{code}
\end{frame}

%%%%%%%%%%%%%%%%%%%%%%%%%%%%%%%%%%%%%%%%%%%%%%%%%%%%%%%%%%%%%%%%%%%%%%

\end{document}
